\documentclass[a4paper]{report}
\usepackage[utf8]{inputenc}
\usepackage[portuguese]{babel}
\usepackage{hyperref}
\usepackage{a4wide}
\hypersetup{pdftitle={CC - TP01},
pdfauthor={João Teixeira, José Ferreira, Miguel Solino},
colorlinks=true,
urlcolor=blue,
linkcolor=black}
\usepackage{subcaption}
\usepackage[cache=false]{minted}
\usepackage{listings}
\usepackage{booktabs}
\usepackage{multirow}
\usepackage{appendix}
\usepackage{tikz}
\usepackage{authblk}
\usepackage{bashful}
\usepackage{verbatim}
\usepackage{amsmath}
\usepackage{tikz}
\usepackage{tikz,fullpage}
\usepackage{pgfgantt}
\usetikzlibrary{arrows,%
                petri,%
                topaths}%
\usepackage{tkz-berge}
\usetikzlibrary{positioning,automata,decorations.markings}
\AfterEndEnvironment{figure}{\noindent\ignorespaces}
\AfterEndEnvironment{table}{\noindent\ignorespaces}

\begin{document}

\title{Comunicação por Computadores\\ 
\large Fase 3 - Grupo 7}
\author{José Ferreira (A83683) \and João Teixeira (A85504) \and Miguel Solino (A86435)}
\date{\today}

\begin{center}
    \begin{minipage}{0.75\linewidth}
        \centering
        \includegraphics[width=0.4\textwidth]{images/eng.jpeg}\par\vspace{1cm}
        \vspace{1.5cm}
        \href{https://www.uminho.pt/PT}
        {\color{black}{\scshape\LARGE Universidade do Minho}} \par
        \vspace{1cm}
        \href{https://www.di.uminho.pt/}
        {\color{black}{\scshape\Large Departamento de Informática}} \par
        \vspace{1.5cm}
        \maketitle
    \end{minipage}
\end{center}

\chapter{Parte 1}
\section{Pergunta a}
\textbf{Qual o conteúdo do ficheiro /etc/resolv.conf e para que serve essa
informação?}
\begin{figure}[H]
    \centering 
    \includegraphics[width=0.7\textwidth]{images/resolv.png}  
    \caption{/etc/resolv.conf}
    \label{fig:resolv}
\end{figure}

O ficheiro /etc/resolv.conf contém as informações relativas aos servidores
de DNS.

\section{Pergunta b}
\textbf{Os servidores www.sapo.pt. e www.yahoo.com. têm endereços IPv6? Se sim,
quais?}

Para sabermos quais são os endereços IPv6 dos servidores, utilizamos o comando
host.

\begin{figure}[H]
    \centering 
    \includegraphics[width=0.7\textwidth]{images/sapopt.png}  
    \caption{sapos.pt}
    \label{fig:sapopt}
\end{figure}

Como podemos reparar na figura \ref{fig:sapopt}, o servidor www.sapo.pt tem o endereço 
IPv6 2001:8a0:2102:c:213:13:146:142,

\begin{figure}[H]
    \centering 
    \includegraphics[width=0.7\textwidth]{images/yahoocom.png}  
    \caption{yahoo.com}
    \label{fig:yahoocom}
\end{figure}

O servidor www.yahoo.com tem o endereço IPv6 2a00:1288:110:1c, como é
possível verificar na figura \ref{fig:yahoocom}.

\section{Pergunta c}
\textbf{Quais os servidores de nomes definidos para os domínios: “uminho.pt.”,
“pt.” e “.”?}
\begin{figure}[H]
    \centering 
    \includegraphics[width=0.5\textwidth]{images/uminhopt.png}  
    \caption{uminho.pt}
    \label{fig:uminhopt}
\end{figure}

Como podemos observar na figura \ref{fig:uminhopt}, para os servidores de nome 
definidos como "uminho.pt" são dns.uminho.pt.

\begin{figure}[H]
    \centering 
    \includegraphics[width=0.5\textwidth]{images/ptponto.png}  
    \caption{pt.}
    \label{fig:ptponto}
\end{figure}

Para os servidores de nome definidos como "pt." são curiosity.dns.pt
(figura \ref{fig:ptponto}).

\begin{figure}[H]
    \centering 
    \includegraphics[width=0.5\textwidth]{images/ponto.png}  
    \caption{.}
    \label{fig:ponto}
\end{figure}

E para os servidores de nome definidos como "." são a.root-servers.net
(figura \ref{fig:ponto}).

\section{Pergunta d}
\textbf{Existe o domínio nice.software.? Será que nice.software. é um host ou um
domínio ?}

\begin{figure}[H]
    \centering 
    \includegraphics[width=\textwidth]{images/nicesoftware.png}  
    \caption{nice.software.}
    \label{fig:nicesoftware}
\end{figure}

Sendo que, como podemos observar na figura \ref{fig:nicesoftware}, através do nslookup
obtemos endereços de IP, então este dominio existe e é um host.

\section{Pergunta e}
\textbf{Qual é o servidor DNS primário definido para o domínio msf.org.? Este
servidor primário (master) aceita queries recursivas? Porquê?}

\begin{figure}[H]
    \centering 
    \includegraphics[width=\textwidth]{images/dnsprimario.png}  
    \caption{DNS primário}
    \label{fig:dnsprimario}
\end{figure}

Para obtermos o DNS primário utilizamos o comando nslookup com uma query
do tipo soa. O resultado está apresentado na figura \ref{fig:dnsprimario} e
como podemos ver está definido como ns1.dds.nl.

\begin{figure}[H]
    \centering 
    \includegraphics[width=\textwidth]{images/recursivo.png}  
    \caption{teste de recursividade}
    \label{fig:recursivo}
\end{figure}

Com o intuito de testar a recursividade, voltamos a usar o comando nslookup
mas desta vez com o DNS primário. Na figura \ref{fig:recursivo}, não obtivemos
resposta nem do DNS primário para msf.org nem do uminho.pt para o DNS primário.
No entanto, do msf.org para o DNS primário foi obtida uma resposta. Sendo assim
podemos concluir que este não é recursivo.

\section{Pergunta f}
\textbf{Obtenha uma resposta “autoritativa” para a questão anterior.}

\begin{figure}[H]
    \centering 
    \includegraphics[width=\textwidth]{images/autoritativa.png}  
    \caption{autoritativa}
    \label{fig:autoritativa}
\end{figure}

\section{Pergunta g}
\textbf{Onde são entregues as mensagens de correio eletrónico dirigidas aos
presidentes marcelo@presidencia.pt e bolsonaro@casacivil.gov.br?}

\section{Pergunta h}
\textbf{Que informação é possível obter, via DNS, acerca de whitehouse.gov?}

\section{Pergunta i}
\textbf{Consegue interrogar o DNS sobre o endereço IPv6
2001:690:a00:1036:1113::247 usando algum dos clientes DNS? Que informação
consegue obter? Supondo que teve problemas com esse endereço, consegue obter um
contacto do responsável por esse IPv6?}

\section{Pergunta j}
\textbf{Os secundários usam um mecanismo designado por “Transferência de zona”
para se atualizarem automaticamente a partir do primário, usando os parâmetros
definidos no Record do tipo SOA do domínio. Descreve sucintamente esse mecanismo
com base num exemplo concreto (ex: di.uminho.pt ou o domínio cc.pt que vai ser
criado na topologia virtual).}

\chapter{Parte 2}

\end{document}
