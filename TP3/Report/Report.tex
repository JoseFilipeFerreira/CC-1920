\documentclass[a4paper]{report}
\usepackage[utf8]{inputenc}
\usepackage[portuguese]{babel}
\usepackage{hyperref}
\usepackage{a4wide}
\hypersetup{pdftitle={CC - TP01},
pdfauthor={João Teixeira, José Ferreira, Miguel Solino},
colorlinks=true,
urlcolor=blue,
linkcolor=black}
\usepackage{subcaption}
\usepackage[cache=false]{minted}
\usepackage{listings}
\usepackage{booktabs}
\usepackage{multirow}
\usepackage{appendix}
\usepackage{tikz}
\usepackage{authblk}
\usepackage{bashful}
\usepackage{verbatim}
\usepackage{amsmath}
\usepackage{tikz}
\usepackage{tikz,fullpage}
\usepackage{pgfgantt}
\usetikzlibrary{arrows,%
                petri,%
                topaths}%
\usepackage{tkz-berge}
\usetikzlibrary{positioning,automata,decorations.markings}
\AfterEndEnvironment{figure}{\noindent\ignorespaces}
\AfterEndEnvironment{table}{\noindent\ignorespaces}

\begin{document}

\title{Comunicação por Computadores\\ 
\large Fase 3 - Grupo 7}
\author{José Ferreira (A83683) \and João Teixeira (A85504) \and Miguel Solino (A86435)}
\date{\today}

\begin{center}
    \begin{minipage}{0.75\linewidth}
        \centering
        \includegraphics[width=0.4\textwidth]{images/eng.jpeg}\par\vspace{1cm}
        \vspace{1.5cm}
        \href{https://www.uminho.pt/PT}
        {\color{black}{\scshape\LARGE Universidade do Minho}} \par
        \vspace{1cm}
        \href{https://www.di.uminho.pt/}
        {\color{black}{\scshape\Large Departamento de Informática}} \par
        \vspace{1.5cm}
        \maketitle
    \end{minipage}
\end{center}

\chapter{Parte 1}
\section{Pergunta a}
\textbf{Qual o conteúdo do ficheiro /etc/resolv.conf e para que serve essa
informação?}

\section{Pergunta b}
\textbf{Os servidores www.sapo.pt. e www.yahoo.com. têm endereços IPv6? Se sim,
quais?}

\section{Pergunta c}
\textbf{Quais os servidores de nomes definidos para os domínios: “uminho.pt.”,
“pt.” e “.”?}

\section{Pergunta d}
\textbf{Existe o domínio nice.software.? Será que nice.software. é um host ou um
domínio ?}

\section{Pergunta e}
\textbf{Qual é o servidor DNS primário definido para o domínio msf.org.? Este
servidor primário (master) aceita queries recursivas? Porquê?}

\section{Pergunta f}
\textbf{Obtenha uma resposta “autoritativa” para a questão anterior.}

\section{Pergunta g}
\textbf{Onde são entregues as mensagens de correio eletrónico dirigidas aos
presidentes marcelo@presidencia.pt e bolsonaro@casacivil.gov.br?}

\section{Pergunta h}
\textbf{Que informação é possível obter, via DNS, acerca de whitehouse.gov?}

\section{Pergunta i}
\textbf{Consegue interrogar o DNS sobre o endereço IPv6
2001:690:a00:1036:1113::247 usando algum dos clientes DNS? Que informação
consegue obter? Supondo que teve problemas com esse endereço, consegue obter um
contacto do responsável por esse IPv6?}

\section{Pergunta j}
\textbf{Os secundários usam um mecanismo designado por “Transferência de zona”
para se atualizarem automaticamente a partir do primário, usando os parâmetros
definidos no Record do tipo SOA do domínio. Descreve sucintamente esse mecanismo
com base num exemplo concreto (ex: di.uminho.pt ou o domínio cc.pt que vai ser
criado na topologia virtual).}

\chapter{Parte 2}

\end{document}
